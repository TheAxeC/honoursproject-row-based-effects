\subsection{General reflection}
An important, if not the most important, aspect that I learned, is that there is no shame in asking for help. This is a competency I didn't think I would learn, it is also not a competency I thought was a problem. More specifically, during the honoursproject I realised that it was better to ask a lot of questions and timely tell someone when I'm having issues then to wait. \\
\\
In the proposal, I said that the honoursprogramme provided an excellent opportunity to get involved with research. Looking back, this has become the most important reason for me. Without the honoursprogramme, I would have a different view of what research is like. I also said that I liked the challenge. In the end, the honoursproject was even more challenging than I expected it to be. Instead of doing the honoursprogramme over a period of the entire academic year, I completed it in nearly one semester. Because I really enjoyed the challenge that the honoursprogramme provided me, I decided to take the second honoursproject this year aswell. This means that I will complete the entire honoursprogramme in a single year.

\subsection{Relation between honoursprogramme and degree}
The honoursproject is linked to my master degree. The honoursproject helped me decide what I want to do for my masterthesis. My second honoursproject goes further in the effect handlers project. Both honoursprojects build up towards my masterthesis. This is an interesting result from doing the honoursprojects within the computer science departement. If I didn't do my honoursproject within the computer science departement, I wouldn't be able to use the honourprojects as a build up for the masterthesis. This is also something that wasn't planned in advance.  \\
\\
During my bachelors, I had a course on functional and logical programming. This course was useful since it provided me with knowledge about the functional programming paradigm. Though it wasn't a course, I have done several research internships. The content of these internships were not related to programming languages. However, these internships did prepare me for the honoursproject. It taught me how to work independently. \\
\\
There is also a course on Formal Systems and their Applications (H04H8BE) which revolves around the typical structure and composition of a formal system such as the lambda calculus. It is taught during the second semester. I was planning to take this course at first, since the content seemed relevant to my honoursproject. However, since my honoursproject occured throughout the first semester, I wasn't able to take the course and possibly take advantage of the content. \\
\\
Other than these courses, there weren't any courses which directly link to the honoursproject. However there was a synergy between the honoursproject and the courses I followed. This synergy came from the fact that my specialisation, Artificial Intelligence, is linked to the research group where I'm doing my honoursproject.

\subsection{Workload of the honoursprogramme}
In the proposal of the honoursproject, it was also discussed that doing the entire honoursprogramme during the masters would be difficult. To make my second year more bearable, I chose to spend 70 ECTS credits on courses during my first year instead of the usual 60 ECTS credits. This has to consequence that my first year would be even more difficult compared to the honoursproject added to the usual 60 ECTS credits. \\
\\
My academic year hasn't ended yet, but I can say that it wasn't a mistake to take the 70 ECTS credits. It was a huge workload, but it also had another unintended effect. My time management skills improved significantly. The fact that I'm also taking my second honoursproject this year tells that I'm motivated and able to handle the workload. \\
\\
In general, I found the workload of the honoursprogramme to be comparable to the workload of 9 ECTS credits. 

\subsection{Social relevance}
In the initial proposal, it was said that the honoursproject is socially relevant for multiple reasons. Society continuously demands more, faster and more reliable software. If better tools for programming languages are developed, this demand can be satisfied faster. A type-\&-effect system makes it easier to implement backtracking, non-determinism and more. It also makes it easier to detect bugs within a program. Optimizing a type-\&-effect system makes it more viable to use such a system in commercial software. 
