\documentclass[sigplan,10pt]{acmart}\settopmatter{printfolios=true}
%% For final camera-ready submission
%\documentclass[sigplan,10pt]{acmart}\settopmatter{}

\settopmatter{printacmref=false} % Removes citation information below abstract
\renewcommand\footnotetextcopyrightpermission[1]{} % removes footnote with conference information in first column
\pagestyle{plain} % removes running headers

%% Some recommended packages.
\usepackage{booktabs}   %% For formal tables:
                        %% http://ctan.org/pkg/booktabs
\usepackage{subcaption} %% For complex figures with subfigures/subcaptions
                        %% http://ctan.org/pkg/subcaption

\usepackage{mathpartir}
\usepackage{xspace}
\usepackage{stmaryrd}
\usepackage{listings}
\usepackage{newtxmath}

%% Tikz Needed packages
\usepackage{pgfplots}
\pgfplotsset{width=7cm,compat=1.8}
\usepackage{pgfplotstable}
\renewcommand*{\familydefault}{\sfdefault}


\makeatletter\if@ACM@journal\makeatother
%% Journal information (used by PACMPL format)
%% Supplied to authors by publisher for camera-ready submission
\acmJournal{ICFP 17 student research competition}
\acmVolume{1}
\acmNumber{1}
\acmArticle{1}
\acmYear{2017}
\acmMonth{1}
\acmDOI{10.1145/nnnnnnn.nnnnnnn}
\startPage{1}
\else\makeatother
%% Conference information (used by SIGPLAN proceedings format)
%% Supplied to authors by publisher for camera-ready submission
\acmConference[ICFP 2017 Student Research Competition]{ICFP 2017 Student Research Competition}{September, 2017}{Oxford, UK}
\acmYear{2017}
\acmISBN{978-x-xxxx-xxxx-x/YY/MM}
\acmDOI{10.1145/nnnnnnn.nnnnnnn}
\startPage{1}
\fi

%% Copyright information
%% Supplied to authors (based on authors' rights management selection;
%% see authors.acm.org) by publisher for camera-ready submission
\setcopyright{none}             %% For review submission
%\setcopyright{acmcopyright}
%\setcopyright{acmlicensed}
%\setcopyright{rightsretained}
%\copyrightyear{2017}           %% If different from \acmYear

%% Bibliography style
\bibliographystyle{ACM-Reference-Format}
%% Citation style
%% Note: author/year citations are required for papers published as an
%% issue of PACMPL.
%\citestyle{acmauthoryear}  %% For author/year citations
%\citestyle{acmnumeric}     %% For numeric citations
%\setcitestyle{nosort}      %% With 'acmnumeric', to disable automatic
                            %% sorting of references within a single citation;
                            %% e.g., \cite{Smith99,Carpenter05,Baker12}
                            %% rendered as [14,5,2] rather than [2,5,14].
%\setcitesyle{nocompress}   %% With 'acmnumeric', to disable automatic
                            %% compression of sequential references within a
                            %% single citation;
                            %% e.g., \cite{Baker12,Baker14,Baker16}
                            %% rendered as [2,3,4] rather than [2-4].

\newcommand{\lang}{\textsc{Effy}\xspace}
\newcommand{\eff}{\textsc{Eff}\xspace}
\newcommand{\ocaml}{\textsc{OCaml}\xspace}

% Meta-syntax
\newcommand{\bnfis}{\mathrel{\;{:}{:}\!=}\;}
\newcommand{\bnfor}{\mathrel{\;|\;}}
\newcommand{\defeq}{\mathrel{\;\stackrel{\text{def}}{=}\;}}
\newcommand{\set}[1]{\{ #1 \}}

% General syntactic constructs
\newcommand{\kord}[1]{\mathtt{#1}}
\newcommand{\kop}[1]{\;\mathtt{#1}\;}
\newcommand{\kpre}[1]{\mathtt{#1}\;}
\newcommand{\kpost}[1]{\;\mathtt{#1}}

% Types
\newcommand{\type}[1]{\mathtt{#1}}
\newcommand{\boolty}{\type{bool}}
\newcommand{\intty}{\type{int}}
\newcommand{\hto}{\Rightarrow}
\renewcommand{\C}{\underline{C}}
\newcommand{\D}{\underline{D}}
\newcommand{\dirt}{\Delta}
\newcommand{\sig}{\Sigma}

% Expressions and computations
\newcommand{\funtyped}[2]{\kpre{fun} #1 \T #2 \mapsto}

\newcommand{\call}[3]{{{#1}\,{#2}\,{#3}}}
\newcommand{\case}{\mathop{\text{\texttt{|}}}}
\newcommand{\cont}[2]{(#1.\,#2)}
\newcommand{\const}{\kord{k}}
\newcommand{\fls}{\kord{false}}
\newcommand{\fun}[1]{\kpre{fun} #1 \mapsto}
\newcommand{\handler}[1]{\{ #1 \}}
\newcommand{\conditional}[3]{\kpre{if} #1 \kop{then} #2 \kop{else} #3}
\newcommand{\letin}[1]{\kpre{let} #1 \kop{in}}
\newcommand{\doin}[1]{\kpre{do} #1 \kop{ ; }}
\newcommand{\letrecin}[1]{\kpre{let} \kpre{rec} #1 \kop{in}}
\newcommand{\op}{\kord{Op}}
\newcommand{\ops}{\mathcal{O}}
\newcommand{\ocs}{\mathit{ocs}}
\newcommand{\ocsnil}{\kord{nil}}
\newcommand{\tru}{\kord{true}}
\newcommand{\ret}{\kpre{return}}
\newcommand{\withhandle}[2]{\kpre{handle} #2 \kop{with} #1}
\newcommand{\pure}[1]{\kord{pure } #1  }
\newcommand{\longcases}{\call{\op_1}{y}{k} \mapsto c_{\op_1}, \ldots, \call{\op_n}{x}{k} \mapsto c_{\op_n}}
\newcommand{\shortcases}{[\call{\op}{y}{k} \mapsto c_\op]_{\op \in \ops}}
\newcommand{\longhand}[1][\ret x \mapsto c_r]{\handler{#1, \longcases}}
\newcommand{\shorthand}[1][\ret x \mapsto c_r]{\handler{#1, \shortcases}}

% Type-checking
\newcommand{\row}{\mathrel{;} R}
\newcommand{\ctx}{\Gamma}
\newcommand{\ent}{\vdash}
\newcommand{\T}{\mathrel{:}}
\newcommand{\E}{\mathrel{!}}
\newcommand{\covers}{\mathrel{/}}
\renewcommand{\le}{\leqslant}

% Operational semantics
\newcommand{\eval}{\Downarrow}
\newcommand{\hs}{\mathcal{H}}
\newcommand{\nil}{\emptyset}
\newcommand{\cons}{\mathbin{::}}
\newcommand{\hseval}[1][\hs]{\Downarrow_{#1}}
\newcommand{\getval}[1]{{#1}_{\kord{val}}}
\newcommand{\getop}[1]{{#1}_{\kord{op}}}

\newcommand{\todo}[1]{\textcolor{red}{\textsc{Todo:} #1}}

\begin{document}

\title{A core language with row-based effects for optimised compilation}

\author{Axel Faes}
\affiliation{
  \position{student : undergraduate\\ACM Student Member: 2461936}
  \department{Department of Computer Science}
  \institution{KU Leuven}
}
\email{axel.faes@student.kuleuven.be}

\author{Tom Schrijvers}
\affiliation{
  \position{advisor}
  \department{Department of Computer Science}
  \institution{KU Leuven}
}
\email{tom.schrijvers@kuleuven.be}

%% Keywords
%% comma separated list
\keywords{algebraic effect handler, row based effect, optimised compilation}  %% \keywords is optional

%% Abstract
%% Note: \begin{abstract}...\end{abstract} environment must come
%% before \maketitle command
\begin{abstract}
Algebraic effects and handlers are a very active area of research. An important aspect is the development of an optimising compiler. \eff is an ML-style language with support for effects and forms the testbed for the optimising compiler. However, \eff does not offer explicit typing, which makes it easy for type checking bugs to be introduced during the construction of optimised compilation. This work presents a new core language with row-based effects. The core language is explicitly typed in order to reduce bugs in the optimised compilation.
\end{abstract}

\maketitle

\section{Introduction}

\label{intro}
Algebraic effect handling is a very active area of research. Implementations of algebraic effect handlers are becoming available. Because of this, improving performance is becoming the focus of research. A lot of research focusses on speeding up the runtime performance. However, a runtime penalty still occurs. This happens since handlers or continuations need to be repeatedly copied on the heap. Due to this, we are looking towards type-directed optimised compilation of algebraic effect handlers. We want to remove the handlers such that no copying is required and thus no runtime penalty occurs. \\
\\
In our ongoing research towards type-directed optimised compilation, term rewrite rules and purity aware compilation optimise away most handlers. Term rewrite rules use information of the type-\&-effect system. Term rewrite rules perform two types of actions. They remove handlers and apply effects such that eventually the program does not contain any more handlers. Term rewrite rules can also change the syntactic structure in order to expose more possibilities for optimisations. Purity aware compilation identifies computations that are effectively pure and purifies them.  \\
\\
\eff, an ML-style language, is being used to develop an optimised compiler for algebraic effect handlers. \eff uses a type system based on subtyping \cite{effectsystem}. As explained by Bauer and Pretnar in \cite{programming}, terms in \eff do not contain any information about computational effects. This information has to be inferred using type inference algorithms. The lack of explicit type information makes source-to-source transformations much more error-prone. Additionally, ensuring that a transformation does not break typeability becomes a time-consuming task, since we need to reconstruct types after each optimisation pass. \\
\\
The current type system with subtyping becomes impractical since the typing information is not explicitly contained in each term. There are several solutions to make the type system more practical. It is possible to keep subtyping, but use a unification based algorithm \cite{mlsub}. Implicit effect polymorphism can also be used \cite{impliciteff}. The option that is explored in this work, is to use a simple type-\&-effect system based on row-polymorphism \cite{type-directed, leijen2014koka, row}. \\
\\
In this work, we present a simple explicitly-typed language that can serve as an intermediate language during compilation of \eff, and allows for the development of type-preserving core-to-core transformations. Optimisation and term rewriting is done using this core language. This approach will ease the development of an optimised compiler since typechecking becomes linear due to the explicit typing.\\
\\
Below is an example program:
\begin{lstlisting}[language=Caml]
effect Op : unit -> int;;
let rec x () = #Op ();;
let result =
  handle (x ()) with
    | #Op () k -> k 1
\end{lstlisting}
This program will be optimised into the code below after applying a function specialization optimization. The function 'x' is specialized and the handler is brought inside the function. Implementing this optimization requires indepth knowledge of the typing system that is used. The subtyping-based approach for the type system is difficult to use and does not contain explicit typing information. This makes source-to-source transformations error prone and
ensuring transformations do not break typability is time consuming.

\begin{lstlisting}[language=Caml]
effect Op : unit -> int;;
let rec x () = #Op ();;
let result =
  let rec x_spec () =
    handle (#Op ()) with
    | #Op () k -> k 1
  in
  x_spec ()
\end{lstlisting}

\\
\\
\todo{showing an example of a bug that can be manifested at an intermediate stage of a compilation and would be caught earlier if only Eff had explicit type annotations.}


\section{Background (\eff)}
The type-\&-effect system that is used in \eff is based on subtyping and dirty types \cite{effectsystem}.

\subsection{Types and terms}

\paragraph{Terms}
Figure~\ref{fig:terms:eff} shows the two types of terms in \eff. There are values $v$ and computations $c$. Computations are terms that can contain effects. Effects are denoted as operations $Op$ which can be called.

\paragraph{Types}
Figure~\ref{fig:types:eff} shows the types of \eff. There are two main sorts of types. There are (pure) types $A, B$ and dirty types $\C, \D$. A dirty type is a pure type $A$ tagged with a finite set of operations $\dirt$, which we call dirt, that can be called. This finite set $\dirt$ is an over-approximation of the operations that are actually called. The type $\C \hto \D$ is used for handlers because a handler takes an input computation $\C$, handles the effects in this computation and outputs computation $\D$ as the result.

\begin{figure}[h]
\begin{center}
\framebox{
\begin{minipage}{0.98\columnwidth}
\[\begin{array}{r@{~}c@{~}l@{\quad}l}
  \text{value}~v & \bnfis {} & x & \text{variable} \\
    & \bnfor & \const & \text{constant} \\
    & \bnfor & \fun{x} c & \text{function} \\
    & \bnfor & \{ & \text{handler} \\
    & & \quad \ret x \mapsto c_r, & \quad\text{return case} \\
    & & \quad \shortcases & \quad\text{operation cases} \\
    & & \} & \\
  \text{comp}~c & \bnfis & v_1 \, v_2 & \text{application} \\
    & \bnfor & \letrecin{f \, x = c_1} c_2 & \text{rec definition} \\
    & \bnfor & \ret v  & \text{returned val} \\
    & \bnfor & \op \, v & \text{operation call} \\
    & \bnfor & \doin{x \leftarrow c_1} c_2 & \text{sequencing} \\
    & \bnfor & \withhandle{v}{c} & \text{handling}
\end{array}\]
\end{minipage}
}
\end{center}
\caption{Terms of \eff}\label{fig:terms:eff}
\end{figure}
\begin{figure}
\begin{center}
\framebox{
\begin{minipage}{0.98\columnwidth}
\[\begin{array}{r@{~}c@{~}l@{\quad}l}
  \text{(pure) type}~A, B & \bnfis {}
    & \boolty \bnfor \intty & \text{basic types} \\
    & \bnfor & A \to \C & \text{function type} \\
    & \bnfor & \C \hto \D & \text{handler type} \\
  \text{dirty type}~\C, \D & \bnfis {} & A \E \dirt \\
  \text{dirt}~\dirt & \bnfis {} &\set{\op_1, \dots, \op_n}
\end{array}\]
\end{minipage}
}
\end{center}
\caption{Types of \eff}\label{fig:types:eff}
\end{figure}

\subsection{Type System}

\subsubsection{Subtyping}
\todo{subtyping}
% The dirty type $A \E \dirt$ is assigned to a computation returning values of type $A$
% and potentially calling operations from the set $\dirt$.
% This set $\dirt$ is always an over-approximation of the actually called operations, and
% may safely be increased, inducing a natural
% subtyping judgement $A \E \dirt \leq A \E \dirt'$ on dirty types. As dirty
% types can occur inside pure types, we also get a derived subtyping judgement on
% pure types. Both judgements are defined in Figure~\ref{fig:subtyping}.
% Observe that, as usual, subtyping is contravariant in the argument types of
% functions and handlers, and covariant in their return types.

\begin{figure}[t!]
\begin{center}
\framebox{
\begin{minipage}{0.95\columnwidth}
\textbf{Subtyping}
\begin{mathpar}
  \inferrule[Sub-$\boolty$]{
  }{
    \boolty \le \boolty
  }

  \inferrule[Sub-$\intty$]{
  }{
    \intty \le \intty
  }

  \inferrule[Sub-$\to$]{
    A' \le A \\
    \C \le \C'
  }{
    A \to \C \le A' \to \C'
  }

  \inferrule[Sub-$\hto$]{
    \C' \le \C \\
    \D \le \D'
  }{
    \C \hto \D \le \C' \hto \D'
  }

  \inferrule[Sub-$\E$]{
    A \le A' \\
    \dirt \subseteq \dirt'
  }{
    A \E \dirt \le A' \E \dirt'
  }
\end{mathpar}
\end{minipage}
}
\end{center}
\caption{Subtyping for pure and dirty types of \eff}\label{fig:subtyping}
\end{figure}

\subsubsection{Typing rules}
Figure~\ref{fig:eff-typing} defines the typing judgements for values and computations with respect to a standard typing context $\ctx$. \\
\\
\todo{values and terms}
% \paragraph{Values}
% The rules for subtyping, variables, and functions are entirely
% standard. For constants we assume a signature $\sig$ that assigns a type~$A$
% to each constant~$\const$, which we write as $(\const \T A) \in \sig$.
%
% A handler expression has type $A \E \dirt \cup \ops \hto B \E \dirt$
% iff all branches (both the operation cases and the return case) have dirty type $B \E \dirt$
% and the operation cases cover the set of operations $\ops$. Note that the intersection $\dirt \cap \ops$ is not necessarily empty. The handler deals with the operations $\ops$, but in the process may re-issue some of them (i.e., $\dirt \cap \ops$).
%
% When typing operation cases, the given signature for the operation $(\op \T A_\op \to B_\op) \in \sig$
% determines the type $A_\op$ of the parameter $x$ and the domain $B_\op$ of the continuation $k$. As our handlers are deep, the codomain of $k$ should be the same as the type $B \E \dirt$ of the cases.
%
% \paragraph{Computations}
% With the following exceptions, the typing judgement $\ctx \ent c : \C$ has a
% straightforward definition. The $\ret$ construct renders a value $v$ as a pure
% computation, i.e., with empty dirt. An operation invocation $\op\,v$ is typed
% according to the operation's signature, with the operation itself as its only
% operation. Finally,
% rule \textsc{With} shows that a handler with type $\C \hto \D$
% transforms a computation with type $\C$ into a computation with type $\D$.

\begin{figure}
\begin{center}
\framebox{
\begin{minipage}{0.95\columnwidth}
\[\begin{array}{r@{~}c@{~}l}
  \text{typing contexts}~\ctx & \bnfis {} & \epsilon \bnfor \ctx, x : A\\
\end{array}\]
\textbf{Expressions}
\begin{mathpar}
  \inferrule[SubVal]{
    \ctx \ent v \T A \\
    A \le A'
  }{
    \ctx \ent v \T A'
  }

  \inferrule[Var]{
    (x \T A) \in \ctx
  }{
    \ctx \ent x \T A
  }

  \inferrule[Const]{
    (\const \T A) \in \sig
  }{
    \ctx \ent \const \T A
  }

  \inferrule[Fun]{
    \ctx, x \T A \ent c \T \C
  }{
    \ctx \ent \fun{x} c \T A \to \C
  }

  \inferrule[Hand]{
    \ctx, x \T A \ent c_r \T B \E \dirt \\
    \Big[
      (\op \T A_\op \to B_\op) \in \sig \qquad \\
      \ctx, x \T A_\op, k \T B_\op \to B \E \dirt \ent c_\op \T B \E \dirt
    \Big]_{\op \in \ops}
  }{
    \ctx \ent \shorthand \T \\ A \E \dirt \cup \ops \hto B \E \dirt
  }
\end{mathpar}
\textbf{Computations}
\begin{mathpar}
  \inferrule[SubComp]{
    \ctx \ent c \T \C \\
    \C \le \C'
  }{
    \ctx \ent c \T \C'
  }

  \inferrule[App]{
    \ctx \ent v_1 \T A \to \C \\
    \ctx \ent v_2 \T A
  }{
    \ctx \ent v_1 \, v_2 \T \C
  }

 \inferrule[LetRec]{
    \ctx, f \T A \to \C, x \T A \ent c_1 \T \C \\
    \ctx, f \T A \to \C \ent c_2 \T \D
  }{
    \ctx \ent \letrecin{f \, x = c_1} c_2 \T \D
  }

  \inferrule[Ret]{
    \ctx \ent v \T A
  }{
    \ctx \ent \ret v \T A \E \emptyset
  }

  \inferrule[Op]{
    (\op \T A \to B) \in \sig \\
    \ctx \ent v \T A
  }{
    \ctx \ent \op \, v \T B \E \{\op\}
  }

  \inferrule[Do]{
    \ctx \ent c_1 \T A \E \dirt \\
    \ctx, x \T A \ent c_2 \T B \E \dirt
  }{
    \ctx \ent \doin{x \leftarrow c_1} c_2 \T B \E \dirt
  }

  \inferrule[With]{
    \ctx \ent v \T \C \hto \D \\
    \ctx \ent c \T \C
  }{
    \ctx \ent \withhandle{v}{c} \T \D
  }
\end{mathpar}
\end{minipage}
}
\end{center}
\caption{Typing of \eff}\label{fig:eff-typing}
\end{figure}


\section{Core language}
The core language with row-based effects is based on the explicitly typed language used in Links \cite{row}. Links uses a row polymorphic type-\&-effect system . The design of their calculus is partially based on the type system used by Pretnar which makes it a suitable candidate for our core language \cite{pretnar2015introduction}. The terms of the core language are seen in Figure~\ref{fig:terms:explicit}, the types are seen in the Figure~\ref{fig:types:explicit}.

\subsection{Types and terms}
The proposed type system uses type application and type abstractions in order to attain explicit typing information. The subtyping approach is replaced by polymorphism. Aside from this, the type system remains as close as possible to the source language of Eff in order to maximise compatability.

\begin{figure}[h]
\begin{center}
\framebox{
\begin{minipage}{0.98\columnwidth}
\[\begin{array}{r@{~}c@{~}l@{\quad}l}
  \text{value}~v & \bnfis {} & x & \text{variable} \\
    & \bnfor & \const & \text{constant} \\
    & \bnfor & \lambda (x : A). c & \text{\textbf{function}} \\
    & \bnfor & \Lambda \alpha . c & \text{\textbf{type abstraction}} \\
    & \bnfor & \{ & \text{handler} \\
    & & \quad \ret x \mapsto c_r, & \quad\text{return case} \\
    & & \quad \shortcases & \quad\text{operation cases} \\
    & & \} & \\
  \text{comp}~c & \bnfis & v_1 \, v_2 & \text{application} \\
    & \bnfor & v \, A & \text{\textbf{type application}} \\
    & \bnfor & \letrecin{f \, x = c_1} c_2 & \text{rec definition} \\
    & \bnfor & \ret v  & \text{returned val} \\
    & \bnfor & \op \, v & \text{operation call} \\
    & \bnfor & \doin{x \leftarrow c_1} c_2 & \text{sequencing} \\
    & \bnfor & \withhandle{v}{c} & \text{handling}
\end{array}\]
\end{minipage}
}
\end{center}
\caption{Terms of the explicitly typed core language}\label{fig:terms:explicit}
\end{figure}

\begin{figure}
\begin{center}
\framebox{
\begin{minipage}{0.98\columnwidth}
\[\begin{array}{r@{~}c@{~}l@{\quad}l}
  \text{(pure) type}~A, B & \bnfis {}
    & A \to \C & \text{function type} \\
    & \bnfor & \C \hto \D & \text{handler type} \\
    & \bnfor & \alpha & \text{\textbf{type variable}} \\
    & \bnfor & \forall \alpha . \C & \text{\textbf{polytype}} \\
  \text{dirty type}~\C, \D & \bnfis {} & A \E \dirt \\
  \text{dirt}~\dirt & \bnfis {} & \{\text{R}\} \\
  \text{R} & \bnfis {}
    & \op \row & \text{row} \\
    & \bnfor & \delta & \text{row variable} \\
    & \bnfor & . & \text{end of row}
\end{array}\]
\end{minipage}
}
\end{center}
\caption{Types of the explicitly type core language}\label{fig:types:explicit}
\end{figure}

\subsection{Typing rules}
There are no surprises in the typing rules either. The typing rules are standard rules. It is important to note the row polymorphism in the \textit{Hand} rule. 

\begin{figure}
\begin{center}
\framebox{
\begin{minipage}{0.95\columnwidth}
\[\begin{array}{r@{~}c@{~}l}
  \text{typing contexts}~\ctx & \bnfis {} & \epsilon \bnfor \ctx, x : A\\
\end{array}\]
\textbf{Expressions}
\begin{mathpar}
  \inferrule[Val]{
  }{
    \ctx \ent v \T A
  }

  \inferrule[Var]{
    (x \T A) \in \ctx
  }{
    \ctx \ent x \T A
  }

  \inferrule[Const]{
    (\const \T A) \in \sig
  }{
    \ctx \ent \const \T A
  }

  \inferrule[Fun]{
    \ctx, x \T A \ent c \T \C
  }{
    \ctx \ent \lambda(x \T A). c \T A \to \C
  }

  \inferrule[Type Abstraction]{
    \ctx, \alpha \ent c \T \C
  }{
    \ctx \ent \Lambda \alpha . c \T \forall \alpha . \C
  }

  \inferrule[Hand]{
  	\C = A \E \{Op_i \row\} \\
  	\D = B \E \{Op_i \row\} \\
  	(\op_i \T A_\op \to B_\op) \in \sig \qquad \\
  	h = \shorthand \\
  	\ctx, x \T A_\op \ent c_r \T \D \\
  	\ctx, y \T A_\op, k \T B_\op \to \D \ent c_{op} \T \D \\
  }{
  	\ctx \ent h \T \C \hto \D
  }
\end{mathpar}
\textbf{Computations}
\begin{mathpar}
  \inferrule[Comp]{
  }{
    \ctx \ent c \T \C
  }

  \inferrule[App]{
    \ctx \ent v_1 \T A \to \C \\
    \ctx \ent v_2 \T A
  }{
    \ctx \ent v_1 \, v_2 \T \C
  }

  \inferrule[Type App]{
    \ctx \ent v \T \forall \alpha . \C
  }{
    \ctx \ent v \, A \T \C[A/\alpha]
  }

 \inferrule[LetRec]{
    \ctx, f \T A \to \C, x \T A \ent c_1 \T \C \\
    \ctx, f \T A \to \C \ent c_2 \T \D
  }{
    \ctx \ent \letrecin{f \, x = c_1} c_2 \T \D
  }

  \inferrule[Ret]{
    \ctx \ent v \T A
  }{
    \ctx \ent \ret v \T A \E \emptyset
  }

  \inferrule[Op]{
    (\op \T A \to B) \in \sig \\
    \ctx \ent v \T A \\
    \C \T B \E \{\op \row\}
  }{
    \ctx \ent \op \, v \T \C
  }

  \inferrule[Do]{
    \ctx \ent c_1 \T A \E \dirt \\
    \ctx, x \T A \ent c_2 \T B \E \dirt
  }{
    \ctx \ent \doin{x \leftarrow c_1} c_2 \T B \E \dirt
  }

  \inferrule[With]{
    \ctx \ent v \T \C \hto \D \\
    \ctx \ent c \T \C
  }{
    \ctx \ent \withhandle{v}{c} \T \D
  }
\end{mathpar}
\end{minipage}
}
\end{center}
\caption{Typing of the explicitly typed language}\label{fig:core-typing}
\end{figure}

\subsection{Big-Step Operational Semantics}
\todo{operational semantics}

\begin{figure}
\begin{center}
\framebox{
\begin{minipage}{0.95\columnwidth}
\[
  \text{result}~r \bnfis {}
    \ret v \bnfor
    \call{\op}{v}{\cont{y}{c}}
\]
\textbf{Evaluation}
\begin{mathpar}
  \inferrule[Eval-App]{
    c[v / x] \eval r
  }{
    (\fun{x} c) \, v \eval r
  }

  \inferrule[Eval-LetRec]{
    c_2[(\fun{x} \letrecin{f \, x = c_1} c_1) / f] \eval r
  }{
    \letrecin{f \, x = c_1} c_2 \eval r
  }

  \inferrule[Eval-Ret]{
  }{
    \ret v \eval \ret v
  }

  \inferrule[Eval-Op]{
  }{
    \op \, v \eval \call{\op}{v}{\cont{y}{\ret y}}
  }

  \inferrule[Eval-Do-Ret]{
    c_1 \eval \ret v \\
    c_2[v / x] \eval r
  }{
    \doin{x \leftarrow c_1} c_2 \eval r
  }

  \inferrule[Eval-Do-Op]{
    c_1 \eval \call{\op}{v}{\cont{y}{c_1'}}
  }{
    \doin{x \leftarrow c_1} c_2 \eval \call{\op}{v}{\cont{y}{\doin{x \leftarrow c_1'} c_2}}
  }

  \inferrule[Eval-With-Ret]{
    c \eval \ret v \\
    c_r[v / x] \eval r
  }{
    \withhandle{h}{c} \eval r
  }

  \inferrule[Eval-With-Handled-Op]{
    c \eval \call{\op}{v}{\cont{y}{c'}} \\
    c_\op[v / x, (\fun{y} \withhandle{h}{c'}) / k] \eval r
  }{
    \withhandle{h}{c} \eval r
  }

  \inferrule[Eval-With-Unhandled-Op]{
    c \eval \call{\op'}{v}{\cont{y}{c'}} \\
    \op' \not\in \ops
  }{
    \withhandle{h}{c} \eval \call{\op'}{v}{\cont{y}{\withhandle{h}{c'}}}
  }

%   \inferrule[Eval-If-True]{
%     c_1 \eval r
%   }{
%     \conditional{\tru}{c_1}{c_2} \eval r
%   }
%
%   \inferrule[Eval-If-False]{
%     c_2 \eval r
%   }{
%     \conditional{\fls}{c_1}{c_2} \eval r
%   }
\end{mathpar}
\end{minipage}
}
\end{center}
\caption{Operational semantics (in the last three rules, $h = \shorthand$)}\label{fig:semantics}
\end{figure}
%
% Figure~\ref{fig:semantics} defines the big-step operational
% semantics of \lang. The judgement $c \eval r$
% states that computation $c$ reduces to result $r$. A result is either a returned value,
% $\ret\,v$, or an unhandled operation, $\op\,v\,(y. c)$, where $v$ is the operation's
% parameter and $y.c$ is its continuation.
%
% The rules \textsc{Eval-App} and \textsc{Eval-LetRec} are straightforward.
% Next, the value result is generated by the $\ret\,v$ computation (\textsc{Eval-Ret}), while the unhandled
% operation (with trivial continuation $y.\ret\,y$) is generated by the $\op\,v$
% computation (\textsc{Eval-Op}). If the intermediate result of a sequential $\kord{do}$ is a value (\textsc{Eval-Do-Ret}),
% it is substituted into the second computation. If it is an unhandled operation (\textsc{Eval-Do-Op}),
% the second computation is appended to its continuation.  When a handled
% computation evaluates to a value (\textsc{Eval-With-Ret}), this value is substituted into the handler's
% return case. Finally, an unhandled operation is passed to
% the appropriate operation case, if there is one (\textsc{Eval-With-Handled-Op}), or propagated further, if there
% is not (\textsc{Eval-With-Unhandled-Op}). In either case, the continuation $y.c'$ is handled by the same handler.


\section{Elaboration}
\todo{elaboration}


\begin{figure}
\begin{center}
\framebox{
\begin{minipage}{0.95\columnwidth}
\[\begin{array}{r@{~}c@{~}l}
  \text{typing contexts}~\ctx & \bnfis {} & \epsilon \bnfor \ctx, x : A\\
\end{array}\]
\textbf{Expressions}
\begin{mathpar}
  \inferrule[Val]{
  }{
    \ctx \ent v \T A \leadsto v'
  }

  \inferrule[Var]{
    (x \T S) \in \ctx \\
    S = \forall \bar{\alpha} . A
  }{
    \ctx \ent x \T A[\bar{S}/\bar{\alpha}] \leadsto x \, \bar{S'}
  }

  \inferrule[Const]{
    (\const \T A) \in \sig
  }{
    \ctx \ent \const \T A \leadsto \const'
  }

  \inferrule[Fun]{
    \ctx, x \T A \ent c \T \C \leadsto c'
  }{
    \ctx \ent \fun{x} c \T A \to \C \leadsto \lambda (x \T A) . c' \T A \to \C
  }

  \inferrule[Hand]{
    \ctx, x \T A \ent c_r \T B \E \dirt \\
    \Big[
      (\op \T A_\op \to B_\op) \in \sig \qquad \\
      \ctx, x \T A_\op, k \T B_\op \to B \E \dirt \ent c_\op \T B \E \dirt
    \Big]_{\op \in \ops}
  }{
    \ctx \ent \shorthand \T \\ A \E \dirt \cup \ops \hto B \E \dirt
  }

  \inferrule[Hand]{
  	\C = A \E \{Op_i \row\} \\
  	\D = B \E \{Op_i \row\} \\
  	(\op_i \T A_\op \to B_\op) \in \sig \qquad \\
  	h = \shorthand \\ \leadsto h' = \shorthandelab \\
  	\ctx, x \T A_\op \ent c_r \T \D \leadsto c'_r \T \D \\
  	\ctx, y \T A_\op, k \T B_\op \to \D \ent c_{op} \T \D \leadsto c'_{op} \T \D\\
  }{
  	\ctx \ent h \T \C \hto \D \leadsto h' \T \C \hto \D
  }
\end{mathpar}
\end{minipage}
}
\end{center}
\caption{Elaboration of source to core language: expressions}\label{fig:elaboration}
\end{figure}


\begin{figure}
\begin{center}
\framebox{
\begin{minipage}{0.95\columnwidth}
\[\begin{array}{r@{~}c@{~}l}
  \text{typing contexts}~\ctx & \bnfis {} & \epsilon \bnfor \ctx, x : A\\
\end{array}\]
\textbf{Computations}
\begin{mathpar}
  \inferrule[Comp]{
  }{
    \ctx \ent c \T \C' \leadsto c'
  }

  \inferrule[App]{
    \ctx \ent v_1 \T A \to \C \leadsto v'_1 \\
    \ctx \ent v_2 \T A \leadsto v'_2
  }{
    \ctx \ent v_1 \, v_2 \T \C \leadsto v'_1 \, v'_2 \T \C
  }

 \inferrule[LetRec]{
    \ctx, f \T A \to \C, x \T A \ent c_1 \T \C \leadsto c'_1 s\\
    \ctx, f \T A \to \C \ent c_2 \T \D \leadsto c'_2
  }{
    \ctx \ent \letrecin{f \, x = c_1} c_2 \T \D \\ \leadsto \letrecin{f \, x = c'_1} c'_2 \T \D
  }

  \inferrule[Ret]{
    \ctx \ent v \T A \leadsto v'
  }{
    \ctx \ent \ret v \T A \E \emptyset \leadsto \ret v' \T A \E \emptyset
  }

  \inferrule[Op]{
    (\op \T A \to B) \in \sig \\
    \C = B \E \{\op \row \} \\
    \ctx \ent v \T A \leadsto v'
  }{
    \ctx \ent \op \, v \T \C \leadsto \op \, v' \T \C
  }

  \inferrule[Do]{
    \ctx \ent c_1 \T \C \leadsto c'_1 \\
    S = \forall \bar{\alpha} . A \\
    \bar{\alpha} = FTV(A) - TV(\ctx) \\
    \ctx, x \T S \ent \D \leadsto c'_2
  }{
    \ctx \ent \doin{x \leftarrow c_1} c_2 \T \D \leadsto (\lambda (x \T A) . c'_2) (\Lambda \bar{\alpha} . c'_1)
  }

  \inferrule[With]{
    \ctx \ent v \T \C \hto \D \leadsto v' \\
    \ctx \ent c \T \C \leadsto c'
  }{
    \ctx \ent \withhandle{v}{c} \T \D \leadsto \withhandle{v'}{c'} \T \D
  }
\end{mathpar}
\end{minipage}
}
\end{center}
\caption{Elaboration of source to core language: computations}\label{fig:elaboration}
\end{figure}


\section{Proofs}
\todo{proofs}


\section{Conclusion}
\subsection{Looking back}
I was able to accomplish more than I expected in the honoursproject. I expected that I could only implement one optimization and that by then I would be done. Multiple optimizations were implemented. In this regard, I underestimated the amount of work I could do in the honoursproject. \\
\\
However, It took longer to orientate myself in the landscape of algebraic effect handlers than I expected. I found that there is a big difference between getting familiar enough with algebraic effects and handlers so I could use them within programs and getting familiar enough with algebraic effects and handlers so I could do research within that field. More specifically, I could write small programs that use algebraic effects and handlers quite fast, but it took a lot longer to be able to do the optimizations. 

\subsection{Where the goals reached}
In the motivation letter, multiple steps were given that needed to be accomplished.
\begin{enumerate}
\item Literature study of Eff and existing optimizations \label{literature}
\item Designing new optimizations \label{design}
\item implementation \label{implementation}
\item evaluation through benchmarks \label{evaluation}
\item formal proof of the optimizations \label{proof}
\end{enumerate}
These steps, with the exclusion of step~\ref{proof}, were accomplished. In the group, it was decided that it was best for me to continue working on the benchmarks and writing the paper instead of writing formal proofs. \\
\\
Step~\ref{literature} was the first task that was accomplished. This was essential for me to be able to contribute to the research project. Afterwards step~\ref{design} and step~\ref{implementation} were done in parallel. Finally step~\ref{evaluation} and working on the paper were done. 

\subsection{Value of the Honoursprogramme}
Other goals that are important for the honoursprogramme concern my own contributions and the knowledge and experience that were gained. During the honoursproject, I was partly treated as an honoursstudent and partly as a researcher. I was partly treated as an honourstudent since I did have regular meetings with professor Schrijvers. But I was also treated as a researcher as I did have a voice during meetings for the project. This meant that I was able to make contributions and state my opinion about decisions before they are made. Ofcourse this came with the responsibility that I carried my own weight.\\
\\
I would say there were a lot of values to the honoursproject. Some I have already mentioned. I got introduced to being a researcher. The honoursproject was quite a challenge, which I definitely found a value of the honoursproject.  I believe that because of these values, my thirst for more knowledge also increased. I want to delve deeper into type-\&-effect systems, learn more about the theoretical foundations. 

\subsection{Conclusion}
To conclude, something that definitely needs to be said is that I'm proud of what I've accomplished during this honoursproject. Choosing to participate in the honoursprogramme and doing my honoursproject with professor Schrijvers were the best choices I could have made. The honoursproject taught me a lot about research, programming language theory and myself. 


%% Acknowledgments
\begin{acks}
  I would like to thank Amr Hany Saleh for his continuous guidance and help. I would also like to thank Matija Pretnar for his support during my research.
\end{acks}

\bibliography{bib/main}

\end{document}
